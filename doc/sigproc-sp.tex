% THIS IS SIGPROC-SP.TEX - VERSION 3.1
% WORKS WITH V3.2SP OF ACM_PROC_ARTICLE-SP.CLS
% APRIL 2009
%
% It is an example file showing how to use the 'acm_proc_article-sp.cls' V3.2SP
% LaTeX2e document class file for Conference Proceedings submissions.
% ----------------------------------------------------------------------------------------------------------------
% This .tex file (and associated .cls V3.2SP) *DOES NOT* produce:
%       1) The Permission Statement
%       2) The Conference (location) Info information
%       3) The Copyright Line with ACM data
%       4) Page numbering
% ---------------------------------------------------------------------------------------------------------------
% It is an example which *does* use the .bib file (from which the .bbl file
% is produced).
% REMEMBER HOWEVER: After having produced the .bbl file,
% and prior to final submission,
% you need to 'insert'  your .bbl file into your source .tex file so as to provide
% ONE 'self-contained' source file.
%
% Questions regarding SIGS should be sent to
% Adrienne Griscti ---> griscti@acm.org
%
% Questions/suggestions regarding the guidelines, .tex and .cls files, etc. to
% Gerald Murray ---> murray@hq.acm.org
%
% For tracking purposes - this is V3.1SP - APRIL 2009

\documentclass{acm_proc_article-sp}
\usepackage{setspace}
\usepackage{xeCJK}
\usepackage{color}
\usepackage{etoolbox}
\usepackage{adjustbox}
\usepackage{array}
\usepackage{threeparttable}
\usepackage{booktabs}
\usepackage{mathtools}
\usepackage{listings}          % format code
\usepackage{url}
\usepackage{tabularx}
\usepackage{multirow}
\usepackage{algorithm2e}
\setCJKmainfont[BoldFont=SimHei]{SimSun}
\setCJKfamilyfont{hei}{SimHei}
\setCJKfamilyfont{kai}{KaiTi}
\setCJKfamilyfont{fang}{FangSong}
\newcommand{\hei}{\CJKfamily{hei}}
\newcommand{\kai}{\CJKfamily{kai}}
\newcommand{\fang}{\CJKfamily{fang}}

\newcommand{\chuhao}{\fontsize{42pt}{\baselineskip}\selectfont}

\newcommand{\xiaochuhao}{\fontsize{36pt}{\baselineskip}\selectfont}
\newcommand{\yihao}{\fontsize{28pt}{\baselineskip}\selectfont}
\newcommand{\erhao}{\fontsize{21pt}{\baselineskip}\selectfont}
\newcommand{\xiaoerhao}{\fontsize{18pt}{\baselineskip}\selectfont}
\newcommand{\sanhao}{\fontsize{15.75pt}{\baselineskip}\selectfont}
\newcommand{\sihao}{\fontsize{14pt}{\baselineskip}\selectfont}
\newcommand{\xiaosihao}{\fontsize{12pt}{\baselineskip}\selectfont}
\newcommand{\wuhao}{\fontsize{10.5pt}{\baselineskip}\selectfont}
\newcommand{\xiaowuhao}{\fontsize{9pt}{\baselineskip}\selectfont}
\newcommand{\liuhao}{\fontsize{7.875pt}{\baselineskip}\selectfont}
\newcommand{\qihao}{\fontsize{5.25pt}{\baselineskip}\selectfont}
\newcommand{\zap}[1]{}
\newcommand\TODO[1]{{\color{red}{Todo: #1}}}
\newcommand{\TBD}[1]{{\color{blue}#1}}

\begin{document}
\begin{spacing}{1.2}
\title{\xiaoerhao\hei 基于天猫交易数据的分析与挖掘\titlenote{具体数据集信息参见https://tianchi.aliyun.com/datalab/d-ataSet.htm?id=5}}
\zap{
\subtitle{[Extended Abstract]
\titlenote{A full version of this paper is available as
\textit{Author's Guide to Preparing ACM SIG Proceedings Using
\LaTeX$2_\epsilon$\ and BibTeX} at
\texttt{www.acm.org/eaddress.htm}}}
}
%
% You need the command \numberofauthors to handle the 'placement
% and alignment' of the authors beneath the title.
%
% For aesthetic reasons, we recommend 'three authors at a time'
% i.e. three 'name/affiliation blocks' be placed beneath the title.
%
% NOTE: You are NOT restricted in how many 'rows' of
% "name/affiliations" may appear. We just ask that you restrict
% the number of 'columns' to three.
%
% Because of the available 'opening page real-estate'
% we ask you to refrain from putting more than six authors
% (two rows with three columns) beneath the article title.
% More than six makes the first-page appear very cluttered indeed.
%
% Use the \alignauthor commands to handle the names
% and affiliations for an 'aesthetic maximum' of six authors.
% Add names, affiliations, addresses for
% the seventh etc. author(s) as the argument for the
% \additionalauthors command.
% These 'additional authors' will be output/set for you
% without further effort on your part as the last section in
% the body of your article BEFORE References or any Appendices.

\numberofauthors{8} %  in this sample file, there are a *total*
% of EIGHT authors. SIX appear on the 'first-page' (for formatting
% reasons) and the remaining two appear in the \additionalauthors section.
%
\begin{spacing}{1.2}
\author{
% You can go ahead and credit any number of authors here,
% e.g. one 'row of three' or two rows (consisting of one row of three
% and a second row of one, two or three).
%
% The command \alignauthor (no curly braces needed) should
% precede each author name, affiliation/snail-mail address and
% e-mail address. Additionally, tag each line of
% affiliation/address with \affaddr, and tag the
% e-mail address with \email.
%
% 1st. author
\alignauthor
\wuhao\kai {\xiaosihao 王希梅}\\
       \affaddr{清华大学\ 软件学院}\\
       \affaddr{北京 \wuhao 100084}\\
       \email{\wuhao wxm17@mails.tsinghua.edu.cn}
% 2nd. author
\alignauthor
\wuhao\kai {\xiaosihao 于千山}\\
       \affaddr{清华大学\ 软件学院}\\
       \affaddr{北京 \wuhao 100084}\\
       \email{\wuhao yqs17@mails.tsinghua.edu.cn}
% 3rd. author
\alignauthor
\wuhao\kai {\xiaosihao 陳善宇}\\
       \affaddr{清华大学\ 软件学院}\\
       \affaddr{北京 \wuhao 100084}\\
       \email{\wuhao abc321094@gmail.com}
%\and  % use '\and' if you need 'another row' of author names
}
\end{spacing}
% There's nothing stopping you putting the seventh, eighth, etc.
% author on the opening page (as the 'third row') but we ask,
% for aesthetic reasons that you place these 'additional authors'
% in the \additional authors block, viz.

% Just remember to make sure that the TOTAL number of authors
% is the number that will appear on the first page PLUS the
% number that will appear in the \additionalauthors section.

\maketitle
\begin{abstract}
我们利用2014年双11(11月11日)前六个月天猫的用户行为日志进行数据挖掘任务。在本次任务中,我们提出四个问题,并分别设计算法进行解决,同时给出了实验结果并进行讨论。这几个问题的解决,对于商家降低促销成本,提高投资回报率(ROI)非常重要。
\end{abstract}

\zap{
% A category with the (minimum) three required fields
\category{H.4}{Information Systems Applications}{Miscellaneous}
%A category including the fourth, optional field follows...
\category{D.2.8}{Software Engineering}{Metrics}[complexity measures, performance measures]
}


\keywords{数据挖掘,聚类分析,行为预测} % NOT required for Proceedings

\section{问题描述} % 阐述问题的合理性或价值
\label{sec:problem}
我们提出以下四个问题:
\begin{enumerate}
  \item 商品平凡模式挖掘:即分析哪些商品会被一起购买。
  \item 同一商家用户的分类:商家会拥有很多用户,对于商家来说,如果能够把握用户的行为,清楚用户的类别,就可以更好地提供服务、贩卖商品,并提高投资回报率。在该问题中,我们考虑将同一商家的用户分为\emph{忠实客户}、\emph{潜在忠实客户}和\emph{非忠实客户}三类。
  \item 同一商家常被购买物品\TBD{种类\ or\ 某些特定物品\  前者意味着还要对物品进行特征提取和分类}:通过对用户行为日志的进行分析,抽取出用户喜好的商品种类。对于商家来说,这有助于把握用户喜好,以便于进行针对性促销。
  \item 用户重复购买的预测:为了吸引大量的新买家,商家有时会在特定的日子进行大促销)。然而,很多买家都是一次性买家,这些促销活动可能对销售的影响不大。我们期望预测未来哪些特定商家的新买家将成为忠实客户。这些新买家将来再次购买同一批商品的可能性。
\end{enumerate}

\section{方法设计}
这一节中,我们考虑对第~\ref{sec:problem}节提出的问题进行方法设计。
\subsection{商品平凡模式挖掘}
我们使用FPGrowth算法进行商品关联规则挖掘,

该算法...\TODO{算法review}

数据...\TODO{描述使用到的数据}

\subsection{同一商家用户的分类}
我们要将同一商家的用户分为\emph{忠实客户}、\emph{潜在忠实客户}和\emph{非忠实客户}三类,首先需要对用户进行特征提取。我们考虑以下特征:\TODO{给一个table}

\subsection{同一商家常被购买物品}
\TBD{如果不考虑分类就只需要对同一商家所有用户发生四种操作(要加权,例如“购买”权重最高)的物品进行计数,选出最高的几个即可。}

\subsection{用户重复购买的预测}
%\texttt{{\char'134}user\_gender}
我们提取了 \texttt{user\_gender} 等...个特征...\TODO{给一个table}

然后选用\TBD{几种模型},使用\texttt{train\_format1.csv}中$80\%$的数据训练网络,另外$20\%$的数据进行预测。

\TODO{特征工程}

\TODO{算法框架}

\TODO{模型、融合}


\section{结果与分析}
我们所有的实验均在\TODO{实验环境}的主机上进行运行。
\begin{itemize}
  \item \textbf{CPU:} 2.3GHz Xeon E5 CPU 72 Cores
  \item \textbf{RAM:} 250GB
  \item \textbf{OS:} \TBD{...}
  \item \textbf{Kernel:} \TBD{...}
  \item \textbf{Python:} \TBD{...}
\end{itemize}
\TODO{我们得到如下结果}

\section{总结}
\TBD{本文我们用...数据,...方法,分析了...问题,得到了...结果}
%\end{document}  % This is where a 'short' article might terminate

%ACKNOWLEDGMENTS are optional
%
% The following two commands are all you need in the
% initial runs of your .tex file to
% produce the bibliography for the citations in your paper.
\begin{spacing}{1}
\bibliographystyle{abbrv}
\bibliography{sigproc}  % sigproc.bib is the name of the Bibliography in this case
\end{spacing}
% You must have a proper ".bib" file
%  and remember to run:
% latex bibtex latex latex
% to resolve all references
%
% ACM needs 'a single self-contained file'!
%
%APPENDICES are optional
%\balancecolumns
\end{spacing}
\end{document}
